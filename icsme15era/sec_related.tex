


%
%  Describe deep learning for natural language
%
Statistical natural language models, such as the n-gram model, have
seen widespread use for a variety of natural language processing tasks
due to their simplicity and effectiveness when trained with a
substantial corpus of text.  Software engineering researchers have
shown n-gram models to be even more effective for source code than for
natural language documents~\cite{hindle_naturalness_2012}. 


Recent research in natural language modeling has introduced deep learning
methods, consisting of specific classes and architectures of neural networks,
that can produce capable statistical models of natural language text able to
capture more complex patterns while being trained using smaller corpora relative
to the n-gram model~\cite{mikolov_distributed_2013,le_distributed_2014}. Using
a set of optimizations strategies such models can be built at reasonable
computational costs.  In this paper we examine the effectiveness and potential
applicability of these deep learning models for the problem of feature location
in software engineering.


%  ii.) What is related work using deep learnign in SE?
%    in S.E. White's paper

White et al.~\cite{white_toward_2015} reported promising results in applying
deep learning models for source code, outperforming models based on similar
n-gram configurations on code completion tasks. This work establishes the use of
deep learning for software engineering problems, sketching out several avenues
for future use of such models, including code completion and the building of
synonym dictionaries, among others. Our work explores these models for the
feature location problem during software maintenance. A notable difference to
the prior experiments conducted by White et al. is that the deep models in this
paper are based solely on terms found in identifiers and comments in the source
code of a single software project, while White et al. considered a larger corpus
consisting of the entire source listing of several combined projects.


%  iii.) Context based approaches to feature location
%	*Emily Hill et al. “One the use of Positional Proximity for IR-Based Feature Location”
%       *Shepherd
%       *SWUM
%       *Naturalness

Previous feature location techniques that incorporated surrounding context in
the model have shown high degrees of effectiveness. Examples include the use of
verb and direct object pairs~\cite{shepherd_using_2007} and statement level
markov random fields~\cite{hill_use_2014}. Deep learning approaches allow the
inclusion of broader context than these previous approaches.

Howard at al.~\cite{howard_automatically_2013} utilized rules based on natural
language text in leading method comments to mine synonym dictionaries from
source code repositories. One of the additional capability of deep language
models is the ability to mine such terms at a greater scale relative to the
their approach.
