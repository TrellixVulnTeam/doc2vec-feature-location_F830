
% 
%  ii.) What is feature location? Breifly, what is the feature
%  Location Current State of the Art / Practice *mainly bag of words
%  approaches
%
When starting a maintenance task, software developers commonly need to
locate the relevant features in a potentially large and unfamiliar
code base. Due to the difficulty and importance of this task,
researchers have proposed a number of approaches to improve
developers' effectiveness in locating features, largely based upon
applying natural language analysis techniques to source
code~\cite{dit_feature_2013}. Most of the proposed feature location
techniques have treated source code as an unordered set of natural
language terms, i.e. as a bag-of-words, even though recent fundamental
results have shown that source code contains context and flow that is
even more pronounced than natural language
text~\cite{hindle_naturalness_2012}.


%
%  What are the contributions of this paper and this line of work
%
In this paper, we explore the use of deep learning, which has shown promising
results in modeling natural language, to the task of feature location. In
particular, we investigate the efficacy of document
vectors~\cite{le_distributed_2014}. Deep learning modeling captures the
influence of the surrounding context on each term, which can improve the ranking
of results retrieved for a developer query. Such models encode that in the
statement {\sf diagram.redraw()} the word {\em diagram} is relevant to the word
{\em redraw} and therefore when querying for {\em diagram} program elements
where {\em redraw} is present should also receive a ranking boost.

Deep learning natural language approaches also create a novel notion
of semantic similarity between the source code terms. Semantic
similarity is the result of mapping the corpus terms into a continuous
semantic space, where synonyms, antonyms, and other semantic relations
are encoded and easily composed together.


%
% The results
%

Our preliminary results show that deep learning models of source code provide
improved retrieval results on the Dit et al.~\cite{Dit-etal_2013} dataset.
This dataset contains gold sets extracted the maintenance histories of four open
source projects. We also show that these complex models can be built with
reasonable computational overhead by leveraging recently proposed optimization
approaches. This work establishes several avenues for future exploration and use
of these models to improve developer effectiveness in feature location.



