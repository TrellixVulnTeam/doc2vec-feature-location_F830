\documentclass[conference]{IEEEtran}

\makeatletter
% IEEEtran.cls defines \labelindent for backward compatibility reasons
% Undefine \labelindent to allow the use of package enumitem
\let\labelindent\relax
\makeatother


\usepackage[T1]{fontenc}
\usepackage[para]{footmisc}
\usepackage[pdftex]{graphicx}
\usepackage[utf8]{inputenc}
\usepackage{array}
\usepackage{balance}
\usepackage{booktabs}
\usepackage{cite}
\usepackage{color}
\usepackage{comment}
\usepackage{enumitem}
\usepackage{framed}
\usepackage{listings}
\usepackage{microtype}
\usepackage{subcaption}
\usepackage{url}


% SQUEEZE
%\addtolength{\parskip}{-1pt}


\definecolor{lightred}{RGB}{150,0,0}
\definecolor{lightgreen}{RGB}{0,150,0}
\definecolor{lightblue}{RGB}{0,0,150}

\lstdefinelanguage{diff}{
  morecomment=[f][\color{lightblue}]{diff },
  morecomment=[f][\color{lightblue}]{index },
  morecomment=[f][\color{lightblue}]{@@},     % group identifier
  morecomment=[f][\color{lightred}]-,         % deleted lines
  morecomment=[f][\color{lightgreen}]+,       % added lines
  morecomment=[f][\color{lightblue}]{---},    % Diff header lines (must appear after +,-)
  morecomment=[f][\color{lightblue}]{+++},
}
\hyphenation{}

\newcommand{\attn}[1]{{\color{red}#1}}
\newcommand{\desc}[1]{{\emph{\color{blue}#1}}}
\newcommand{\needcite}{\attn{\tiny{[cite]}}}
\newcommand{\todo}[1]{\strut\smash{\colorbox{yellow}{\bf TODO: #1}}}
\setlength\OuterFrameSep{0.5em}
\setlength\FrameSep{0.5em}

\begin{document}
\title{Exploring the Use of Deep Learning for Feature Location}
\author{
    \IEEEauthorblockN{
        Christopher S.\ Corley
    }
    \IEEEauthorblockA{
        The University of Alabama\\
        Tuscaloosa, AL, USA\\
        cscorley@ua.edu
    }
    \and
    \IEEEauthorblockN{
        Kostadin Damevski
    }
    \IEEEauthorblockA{
	Virginia State University\\
	Petersburg, VA, USA\\
        damevski@acm.org
    }
    \and
    \IEEEauthorblockN{
        Nicholas A.\ Kraft
    }
    \IEEEauthorblockA{
        ABB Corporate Research\\
        Raleigh, NC, USA\\
        nicholas.a.kraft@us.abb.com
    }
}


\maketitle

\begin{abstract}
Deep learning models are a class of neural networks. Relative to n-gram models, deep learning models can capture more complex statistical patterns based on smaller training corpora. In this paper we explore the use of a particular deep learning model, document vectors, for feature location. Document vectors seem well suited to use with source code, because they both capture the influence of context on each term in a corpus and map terms into a continuous semantic space that encodes semantic relationships such as synonymy and polysemy. We present preliminary results that show that a feature location technique (FLT) based on document vectors can outperform an analogous FLT based on latent Dirichlet allocation and then suggest several directions for future work on the use of deep learning models to improve developer effectiveness in feature location.
\end{abstract}

\begin{IEEEkeywords}
feature location;
\end{IEEEkeywords}

\section{Introduction}\label{introduction}

% 
%  ii.) What is feature location? Breifly, what is the feature
%  Location Current State of the Art / Practice *mainly bag of words
%  approaches
%
When starting a maintenance task, software developers commonly need to
locate the relevant features for the task in a potentially large and
unfamiliar code base. Due to the difficulty and importance of this
task, researchers have proposed a number of approaches to improve
developers' effectiveness in locating features, largely based upon
applying natural language analysis techniques to source
code~\cite{dit_feature_2013}. Most of the proposed feature location
techniques have treated source code as an unordered set of natural
language terms, i.e. as a bag-of-words, even though recent fundamental
results have shown that source code contains context and flow that is
even more pronounced than natural language
text~\cite{hindle_naturalness_2012}.


%
%  What are the contributions of this paper and this line of work
%
In this paper, we explore the use of deep learning, which has shown
promising results in modeling natural language, to the task of feature
location. Deep learning captures the influence of the surrounding
context on each term, which can improve the ranking of results
retrieved for a developer query. Such models encode that in the
statement {\sf diagram.redraw()} the word {\em diagram} is relevant
to the word {\em redraw} and therefore when querying for {\em
diagram} program elements where {\em redraw} is present should also
receive a ranking boost.

Deep learning natural language approaches also create a novel notion
of semantic similarity between the source code terms. Semantic
similarity is the results of mapping the corpus terms into a
continuous semantic space, where synonyms, antonyms, and other
semantic relations are encoded and easily composed together.



%
% The results
%

Our preliminary results show that deep learning models of source code
provide improved retrieval results on Y gold sets extracted the
maintenance histories of Z open source projects. We also show that these
complex models can be built with reasonable computational overhead by
leveraging recently proposed optimization approaches. This work
establishes several avenues for future exploration and use of these
models to improve developer effectiveness in feature location.





\section{Related Work}\label{related}



%
%  Describe deep learning for natural language
Statistical natural language models, such as the n-gram model, have
seen widespread use for a variety of natural language processing task
due to their similicity and effectiveness when trained with a
substantial corpus of text. More recent research in this field has
proposed certain classes and architectures of neural networks that can
produce excellent statistical models of natural language text, which
can include deeper contexts and capture more complex patterns, while
trained using smaller corpora, relative to the n-gram model. Using a
set of optimizations strategies such models can be built at reasonable
computational costs for even large textual corpora.


%  ii.) What is related work using deep learnign in SE?
%    in S.E. White's paper





%  iii.) Context based approaches to feature location
%	*Emily Hill et al. “One the use of Positional Proximity for IR-Based Feature Location”
%       *Shepherd
%       *SWUM
%       *Naturalness


hill~\cite{hill_use_2014}


\section{Feature Location Systems Based on Deep Learning}\label{design}

%   i.) Describe all the steps of a Deep Learning FL system
%      
%

%how do FL systems work in general?
Feature location systems retrieve a ranked list of program elements
(e.g. methods, classes) for a developer query. In the {\em indexing}
phase, feature location systems commonly construct a model of the
software, at the granularity of program elements, based on the natural
language embedded in identifiers and comments. In the {\em retrieval}
phase, given a natural language query, feature location systems use
the model to retrieve all of the relevant program elements with high
similarity to the query.

\subsection{Feature Location Workflow}

%how does a deep learning system differ
A feature location system based on deep learning, during its indexing
phase, creates a contextual representation of the natural language
terms embedded in the source code. This contextual representation
includes influence from terms preceding and following each term,
relative to their distance from that term. More intuitively, such
models incoporate mutual influence between terms in the same method,
while terms that are closer in distance (e.g. occur the same
statement) influence each other more strongly. 


%what is deep learning
Deep learning is based on a multi-stage neural network, consisting of
several hidden layers in addition to single input and output layers.
The input layer consists of the ordered sequences of identifiers
extracted from the code. The multiple hidden layers serve to capture
the context for each encountered term, representing the complex
patterns of term contexts occuring in the corpus. The output layer
consists of a vector for each term, which has been shown to carry
semantic meaning. Recent advances in this area have stemmed from the
use of novel neural network architectures, including recurrent neural
networks that connect the hidden layers back to the input layer, among
other strategies. The systems are trained using backpropagation and
gradient descent, techniques common to many neural network based
models.

 
%preprocessing
A number of preprocessing steps are commonly performed during the
indexing phase of feature location systems including word stemming,
which reduces terms to their base form (e.g. running $\rightarrow$
run), and stop word removal, which removes common words (e.g. the, is,
at). Additional preprocessing steps are required for source code, such
as identifier splitting, ensuring that composite identifiers are
divided into their constituent terms (e.g. addToCart $\rightarrow$
add,to,cart). 


During the retrieval state of feature location, a similarity measure
(e.g. cosine similarity) between the words in the query and words in
each program element is computed. The program elements are ranked
based on this similarity metric and presented to the developer in
descending order.



%Chris: I assume this is what gensim is doing, but I am not sure...




\subsection{Semantic Similarity}


\section{Preliminary Results}\label{results}

%     (compare to LDA and VSM)
%     i. effectiveness on gold sets (F-measure; Mean Average Precision & Precision at Rank 1?)
%	*the deep learning works better as the number of terms grows, so make sure those characteristics come out
%
%     ii. speed
%           (for corpus building; for retrieval)


%
% Notes for Chris and Nick:
% 1) would be nice to also see the number of terms and number of unique terms per each project
% 2) i would definitely show both word2vec and doc2vec features and compare to LDA and VSM
% 3) make sure you mention the exact configuration of preprocessing steps (e.g. no word stemming)
%
\subsection{Subject software systems}

All of our subject software systems come from two publicly-available datasets.
The first is a dataset of six software systems by Dit et
al.~\cite{Dit-etal_2013} and contains method-level goldsets.  This dataset was
automatically extracted from changesets that relate to the queries (issue
reports).

\begin{table}[t]
\renewcommand{\arraystretch}{1.3}
\footnotesize
\centering
\caption{Subject System Sizes and Queries}
\begin{tabular}{lrrr}
    \toprule
    Subject System     & Methods & Queries    \\    %& Goldset Methods
    \midrule                                        %
    ArgoUML v0.22      & 12353    & 91        \\    %& 701
    ArgoUML v0.24      & 13064    & 52        \\    %& 357
    ArgoUML v0.26.2    & 16880    & 209       \\    %& 1560
    Jabref v2.6        & 5357     & 39        \\    %& 280
    jEdit v4.3         & 7305     & 150       \\    %& 748
    muCommander v0.8.5 & 8799     & 92        \\    %& 717
    \midrule                                        %
    Total              & 63758    & 633       \\    %& 4363
    \bottomrule
\end{tabular}
\label{table:subjects}
\end{table}


ArgoUML is a UML diagramming tool\footnote{\url{http://argouml.tigris.org/}}.
jEdit is a text editor\footnote{\url{http://www.jedit.org/}}.
JabRef is a BibTeX bibliography management tool\footnote{\url{http://jabref.sourceforge.net/}}.
muCommander is a cross-platform file manager\footnote{\url{http://www.mucommander.com/}}.

\subsection{Methodology}


\subsection{Setting}

\subsection{Data Collection and Analysis}

\subsection{Results}

\begin{table}
    \centering
\begin{tabular}{llccccc}
\toprule
System & Model &     100 &    200 &    300 &    400 &    500 \\
\midrule
                   &             LDA & 0.0175 & 0.0295 & 0.0271 & 0.0611 & 0.0220 \\
ArgoUML v0.22      & D2V Inference   & 0.0115 & 0.0105 & 0.0096 & 0.0184 & 0.0162 \\
                   & D2V Summation   & 0.0775 & 0.0570 & 0.0625 & 0.0587 & 0.0601 \\
                     \midrule
                   &             LDA & 0.0441 & 0.0373 & 0.0655 & 0.0779 & 0.0344 \\
ArgoUML v0.24      & D2V Inference   & 0.0246 & 0.0152 & 0.0260 & 0.0258 & 0.0380 \\
                   & D2V Summation   & 0.0827 & 0.0906 & 0.0874 & 0.0691 & 0.0942 \\
                     \midrule
                   &             LDA & 0.0493 & 0.0628 & 0.0857 & 0.0703 & 0.0811 \\
ArgoUML v0.26.2    & D2V Inference   & 0.0404 & 0.0218 & 0.0290 & 0.0364 & 0.0403 \\
                   & D2V Summation   & 0.0847 & 0.0890 & 0.0813 & 0.0834 & 0.0805 \\
                     \midrule
                   &             LDA & 0.0055 & 0.0364 & 0.1304 & 0.0781 & 0.0548 \\
JabRef v2.6        & D2V Inference   & 0.0262 & 0.0463 & 0.0318 & 0.0289 & 0.0234 \\
                   & D2V Summation   & 0.0450 & 0.0373 & 0.0455 & 0.0382 & 0.0428 \\
                     \midrule
                   &             LDA & 0.0670 & 0.0432 & 0.0641 & 0.0693 & 0.0607 \\
jEdit v4.3         & D2V Inference   & 0.0341 & 0.0282 & 0.0369 & 0.0354 & 0.0450 \\
                   & D2V Summation   & 0.0872 & 0.0791 & 0.0825 & 0.0814 & 0.0679 \\
                     \midrule
                   &             LDA & 0.0392 & 0.0217 & 0.0198 & 0.0559 & 0.0329 \\
muCommander v0.8.5 & D2V Inference   & 0.0977 & 0.0771 & 0.0800 & 0.0665 & 0.0838 \\
                   & D2V Summation   & 0.0652 & 0.0623 & 0.0703 & 0.0606 & 0.0538 \\

\bottomrule
\end{tabular}
\caption{MRR}
\label{tab:mrr}
\end{table}

\begin{table}
\centering
\begin{tabular}{lcc}
\toprule
                {} & LDA    & Doc2Vec \\
\midrule
ArgoUML v0.22      &  0m 58.207s     &  0m 02.070s     \\
ArgoUML v0.24      &  1m 05.507s     &  0m 02.267s     \\
ArgoUML v0.26.2    &  1m 21.176s     &  0m 02.736s     \\
JabRef v2.6        &  0m 29.504s     &  0m 01.280s     \\
jEdit v4.3         &  0m 36.701s     &  0m 01.519s     \\
muCommander v0.8.5 &  0m 42.897s     &  0m 01.696s     \\
\bottomrule
\end{tabular}
\caption{Model training times for 100 topics}
\end{table}


\begin{table}
\centering
\begin{tabular}{lccc}
\toprule
                {} & LDA    & Vec Inference & Vec Sums \\
\midrule
ArgoUML v0.22      & 0.943362s      &  0.225868s     &  2.118835s         \\
ArgoUML v0.24      & 1.686980s      &  0.259615s     &  1.824923s         \\
ArgoUML v0.26.2    & 0.744468s      &  0.300956s     &  2.715062s         \\
JabRef v2.6        & 0.957000s      &  0.124128s     &  0.720589s         \\
jEdit v4.3         & 0.405700s      &  0.134080s     &  1.017060s         \\
muCommander v0.8.5 & 0.685543s      &  0.165663s     &  1.056108s         \\
\bottomrule
\end{tabular}
\caption{Average time to rank per query for 100 topics}
\end{table}



\subsection{Discussion}


% conclusing remarks:
%   doc2vec faster, could be easier to implement directly into an IDE search
%   tool
%   doc2vec inference needs more exploration for parameter tweaking
%   doc2vec vector summation, although slower, is at an acceptable time


\section{Conclusions and Future Work}\label{futurework}

%future work: 
%try in the field (e.g. in Sando)
%maybe try at larger scale 
%incorporate program structure information
%paragraph for each new problem in feature location that this could help (e.g. recommendation)
%Context (compare to n-gram, V-DO stuff)
%


\section*{Acknowledgments}
We acknowledge noone.

\bibliographystyle{IEEEtran}
\bibliography{doc2vec}

\end{document}
