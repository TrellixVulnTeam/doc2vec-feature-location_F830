\documentclass[conference]{IEEEtran}

\makeatletter
% IEEEtran.cls defines \labelindent for backward compatibility reasons
% Undefine \labelindent to allow the use of package enumitem
\let\labelindent\relax
\makeatother


\usepackage[T1]{fontenc}
\usepackage[para]{footmisc}
\usepackage[pdftex]{graphicx}
\usepackage[utf8]{inputenc}
\usepackage{array}
\usepackage{balance}
\usepackage{booktabs}
\usepackage{cite}
\usepackage{color}
\usepackage{comment}
\usepackage{enumitem}
\usepackage{framed}
\usepackage{listings}
\usepackage{microtype}
\usepackage{subcaption}
\usepackage{url}


% SQUEEZE
%\addtolength{\parskip}{-1pt}


\definecolor{lightred}{RGB}{150,0,0}
\definecolor{lightgreen}{RGB}{0,150,0}
\definecolor{lightblue}{RGB}{0,0,150}

\lstdefinelanguage{diff}{
  morecomment=[f][\color{lightblue}]{diff },
  morecomment=[f][\color{lightblue}]{index },
  morecomment=[f][\color{lightblue}]{@@},     % group identifier
  morecomment=[f][\color{lightred}]-,         % deleted lines
  morecomment=[f][\color{lightgreen}]+,       % added lines
  morecomment=[f][\color{lightblue}]{---},    % Diff header lines (must appear after +,-)
  morecomment=[f][\color{lightblue}]{+++},
}
\hyphenation{}

\newcommand{\attn}[1]{{\color{red}#1}}
\newcommand{\desc}[1]{{\emph{\color{blue}#1}}}
\newcommand{\needcite}{\attn{\tiny{[cite]}}}
\newcommand{\todo}[1]{\strut\smash{\colorbox{yellow}{\bf TODO: #1}}}
\setlength\OuterFrameSep{0.5em}
\setlength\FrameSep{0.5em}

\begin{document}
\title{Exploring the Use of Deep Learning for Feature Location in Source Code}
\author{
    \IEEEauthorblockN{
        Christopher S.\ Corley
    }
    \IEEEauthorblockA{
        The University of Alabama\\
        Tuscaloosa, AL, USA\\
        cscorley@ua.edu
    }
    \and
    \IEEEauthorblockN{
        Nicholas A.\ Kraft
    }
    \IEEEauthorblockA{
        ABB Corporate Research\\
        Raleigh, NC, USA\\
        nicholas.a.kraft@us.abb.com
    }
    \and
    \IEEEauthorblockN{
        Kostadin Damevski
    }
    \IEEEauthorblockA{
	Virginia State University\\
	Petersburg, VA, USA\\
        damevski@acm.org
    }
}


\maketitle

\begin{abstract}
\end{abstract}

\begin{IEEEkeywords}
feature location;
\end{IEEEkeywords}

\section{Introduction}
         

\section{Background and Related Work}

%   i.) Deep Learning for NL; and in SE
%


%  ii.) Feature Location Current State of the Art / Practice
%       *bag of words approaches
%  iii.) Context based approaches to feature location
%	*Emily Hill et al. “One the use of Positional Proximity for IR-Based Feature Location”

hill~\cite{hill_use_2014}

\section{Design of Feature Location System Based on Deep Learning}

%   i.) Describe all the steps of the system
%       *is it Sando-like except for the model? (i.e. parsing->stopwords->splitting->indexing+program_element_based_weighting)
%
%  ii.) Describe optimal values for parameters (like window size and feature vector length)
%	*use paper by Poshyvanyk for hints on this
%
%  iii.) What are the expected advantages over its counterparts (n-grams, V-DO, MRF)?
%       *speed and retrieval effectiveness, i suppose
%
%  iv.) Other advantages of deep learning?
%       *e.g. query recommendation based on semantic similarity from deep learning


\section{Preliminary Results}
%approx (2.5 pages)

%     (compare to LDA and VSM)
%     i. effectiveness on gold sets (F-measure; Mean Average Precision & Precision at Rank 1?)
%	*the deep learning works better as the number of terms grows, so make sure those characteristics come out
%
%     ii. speed 
%           (for corpus building; for retrieval)

\section{Future Work and Conclusion}
%(~1 page including references)


%future work: 
%try in the field (e.g. in Sando)
%maybe try at larger scale 
%incorporate program structure information
%paragraph for each new problem in feature location that this could help (e.g. recommendation)
%Context (compare to n-gram, V-DO stuff)



\section*{Acknowledgment}
We acknowledge noone.

\bibliographystyle{IEEEtran}
\bibliography{doc2vec}

\end{document}
