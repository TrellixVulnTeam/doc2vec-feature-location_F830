
%     (compare to LDA and VSM)
%     i. effectiveness on gold sets (F-measure; Mean Average Precision & Precision at Rank 1?)
%	*the deep learning works better as the number of terms grows, so make sure those characteristics come out
%
%     ii. speed 
%           (for corpus building; for retrieval)


%
% Notes for Chris and Nick:
% 1) would be nice to also see the number of terms and number of unique terms per each project
% 2) i would definitely show both word2vec and doc2vec features and compare to LDA and VSM
% 3) make sure you mention the exact configuration of preprocessing steps (e.g. no word stemming)


\begin{table}
    \centering
\begin{tabular}{llccccc}
\toprule
System & Model &     100 &    200 &    300 &    400 &    500 \\
\midrule
                   &             LDA & 0.0175 & 0.0295 & 0.0271 & 0.0611 & 0.0220 \\
ArgoUML v0.22      & D2V Inference   & 0.0115 & 0.0105 & 0.0096 & 0.0184 & 0.0162 \\
                   & D2V Summation   & 0.0775 & 0.0570 & 0.0625 & 0.0587 & 0.0601 \\
                     \midrule
                   &             LDA & 0.0441 & 0.0373 & 0.0655 & 0.0779 & 0.0344 \\
ArgoUML v0.24      & D2V Inference   & 0.0246 & 0.0152 & 0.0260 & 0.0258 & 0.0380 \\
                   & D2V Summation   & 0.0827 & 0.0906 & 0.0874 & 0.0691 & 0.0942 \\
                     \midrule
                   &             LDA & 0.0493 & 0.0628 & 0.0857 & 0.0703 & 0.0811 \\
ArgoUML v0.26.2    & D2V Inference   & 0.0404 & 0.0218 & 0.0290 & 0.0364 & 0.0403 \\
                   & D2V Summation   & 0.0847 & 0.0890 & 0.0813 & 0.0834 & 0.0805 \\
                     \midrule
                   &             LDA & 0.0055 & 0.0364 & 0.1304 & 0.0781 & 0.0548 \\
JabRef v2.6        & D2V Inference   & 0.0262 & 0.0463 & 0.0318 & 0.0289 & 0.0234 \\
                   & D2V Summation   & 0.0450 & 0.0373 & 0.0455 & 0.0382 & 0.0428 \\
                     \midrule
                   &             LDA & 0.0670 & 0.0432 & 0.0641 & 0.0693 & 0.0607 \\
jEdit v4.3         & D2V Inference   & 0.0341 & 0.0282 & 0.0369 & 0.0354 & 0.0450 \\
                   & D2V Summation   & 0.0872 & 0.0791 & 0.0825 & 0.0814 & 0.0679 \\
                     \midrule
                   &             LDA & 0.0392 & 0.0217 & 0.0198 & 0.0559 & 0.0329 \\
muCommander v0.8.5 & D2V Inference   & 0.0977 & 0.0771 & 0.0800 & 0.0665 & 0.0838 \\
                   & D2V Summation   & 0.0652 & 0.0623 & 0.0703 & 0.0606 & 0.0538 \\

\bottomrule
\end{tabular}
\caption{MRR}
\label{tab:mrr}
\end{table}

\begin{table}
\centering
\begin{tabular}{lcc}
\toprule
                {} & LDA    & Doc2Vec \\
\midrule
ArgoUML v0.22      &  0m 58.207s     &  0m 02.070s     \\
ArgoUML v0.24      &  1m 05.507s     &  0m 02.267s     \\
ArgoUML v0.26.2    &  1m 21.176s     &  0m 02.736s     \\
JabRef v2.6        &  0m 29.504s     &  0m 01.280s     \\
jEdit v4.3         &  0m 36.701s     &  0m 01.519s     \\
muCommander v0.8.5 &  0m 42.897s     &  0m 01.696s     \\
\bottomrule
\end{tabular}
\caption{Model training times for 100 topics}
\end{table}


\begin{table}
\centering
\begin{tabular}{lccc}
\toprule
                {} & LDA    & Vec Inference & Vec Sums \\
\midrule
ArgoUML v0.22      & 0.943362s      &  0.225868s     &  2.118835s         \\
ArgoUML v0.24      & 1.686980s      &  0.259615s     &  1.824923s         \\
ArgoUML v0.26.2    & 0.744468s      &  0.300956s     &  2.715062s         \\
JabRef v2.6        & 0.957000s      &  0.124128s     &  0.720589s         \\
jEdit v4.3         & 0.405700s      &  0.134080s     &  1.017060s         \\
muCommander v0.8.5 & 0.685543s      &  0.165663s     &  1.056108s         \\
\bottomrule
\end{tabular}
\caption{Average time to rank per query for 100 topics}
\end{table}


\begin{comment}
    % these are all in mrr_mega now
\begin{table}
\begin{tabular}{lrrrrr}
\toprule
{} &       100 &       200 &       300 &       400 &       500 \\
\midrule
ArgoUML v0.22      &  0.017532 &  0.029460 &  0.027142 &  0.061079 &  0.021953 \\
ArgoUML v0.24      &  0.044053 &  0.037306 &  0.065500 &  0.077855 &  0.034382 \\
ArgoUML v0.26.2    &  0.049304 &  0.062822 &  0.085700 &  0.070299 &  0.081081 \\
JabRef v2.6        &  0.005463 &  0.036370 &  0.130358 &  0.078107 &  0.054799 \\
jEdit v4.3         &  0.067034 &  0.043171 &  0.064146 &  0.069303 &  0.060734 \\
muCommander v0.8.5 &  0.039195 &  0.021695 &  0.019819 &  0.055850 &  0.032859 \\
\bottomrule
\end{tabular}

\caption{MRR of LDA for different number of topics}
\end{table}

\begin{table}
\begin{tabular}{lrrrrr}
\toprule
{} &    100 &    200 &    300 &    400 &    500 \\
\midrule
ArgoUML v0.22      & 0.0115 & 0.0105 & 0.0096 & 0.0184 & 0.0162 \\
ArgoUML v0.24      & 0.0246 & 0.0152 & 0.0260 & 0.0258 & 0.0380 \\
ArgoUML v0.26.2    & 0.0404 & 0.0218 & 0.0290 & 0.0364 & 0.0403 \\
JabRef v2.6        & 0.0262 & 0.0463 & 0.0318 & 0.0289 & 0.0234 \\
jEdit v4.3         & 0.0341 & 0.0282 & 0.0369 & 0.0354 & 0.0450 \\
muCommander v0.8.5 & 0.0977 & 0.0771 & 0.0800 & 0.0665 & 0.0838 \\
\bottomrule
\end{tabular}

\caption{MRR of Doc2Vec inference for different numbers of vector features}
\end{table}

\begin{table}
\begin{tabular}{lrrrrr}
\toprule
{} &       100 &       200 &       300 &       400 &       500 \\
\midrule
ArgoUML v0.22      &  0.077488 &  0.057025 &  0.062467 &  0.058661 &  0.060084 \\
ArgoUML v0.24      &  0.082675 &  0.090578 &  0.087371 &  0.069051 &  0.094173 \\
ArgoUML v0.26.2    &  0.084667 &  0.088990 &  0.081300 &  0.083383 &  0.080502 \\
JabRef v2.6        &  0.044958 &  0.037341 &  0.045498 &  0.038187 &  0.042750 \\
jEdit v4.3         &  0.087213 &  0.079077 &  0.082536 &  0.081364 &  0.067913 \\
muCommander v0.8.5 &  0.065227 &  0.062305 &  0.070269 &  0.060592 &  0.053812 \\
\bottomrule
\end{tabular}

\caption{MRR of Doc2Vec vector summing for different numbers of vector features}
\end{table}
\end{comment}


% conclusing remarks:
%   doc2vec faster, could be easier to implement directly into an IDE search
%   tool
%   doc2vec inference needs more exploration for parameter tweaking
%   doc2vec vector summation, although slower, is at an acceptable time
