
%future work: 
%try in the field (e.g. in Sando)
%maybe try at larger scale 
%incorporate program structure information
%paragraph for each new problem in feature location that this could help (e.g. recommendation)
%Context (compare to n-gram, V-DO stuff)
%

% conclusing remarks:
%   doc2vec faster, could be easier to implement directly into an IDE search
%   tool
%   doc2vec inference needs more exploration for parameter tweaking
%   doc2vec vector summation, although slower, is at an acceptable time


In this work we present a preliminary study of a deep-learning-based
FLT using document vectors (\dv). We find that training the \dv\ 
model has low computational overhead (i.e., is fast), while maintaining
accuracy on par with LDA. We find \dv\ to be a promising solution to implementing
smarter developer search tools in the IDE, a task to which more computationally-intensive models such as LDA are less well-suited.

%Future work includes investigating the approaches performance at differing granularities.
%Additionally, work is needed to investigate effective parameter configurations for the model.
One direction for future work is to explore the effects of parameter tuning on the performance of \dv.
Multiple studies~\cite{Biggers-etal_2014} show that selection of appropriate parameter values is key to the performance of an LDA-based FLT.
Thus, the first question to address is whether the same is true of a \dv-based FLT, and if so, the next question is how to select \dv\ parameters for a particular subject system.
The results of our study show that a \dv-based FLT can provide better accuracy than an LDA-based FLT while requiring fewer computational resources, and more intelligent parameter selection for \dv\ could further improve its accuracy.

Another direction for future work is to use the semantic relations encoded by a \dv\ model for query recommendation or refinement.
We would like to investigate the efficacy of these semantic relations for recommending query terms based on a partial query.
That is, while a developer is entering a query, a recommender could use a \dv\ model of the subject system to recommend additional query terms.
Such an approach would be similar (in function) to the ones implemented in the Sando~\cite{Shepherd:2012} and CONQUER~\cite{CONQUER_2013}.
Similarly, given its relative computational efficiency, using \dv\ as the basis for an IDE-based search tool is a realistic goal.

Further directions for future work include extending the \dv\ model to incorporate program structure information,
% an approach which has shown useful when combined with text retrieval models~\cite{Panichella-etal_2013,Saha-etal_2013,Bassett-Kraft_2013},
an approach which has shown useful when combined with text retrieval models~\cite{Bassett-Kraft_2013},
or extending the \dv\ model to incorporate natural language information such as part-of-speech tags~\cite{shepherd_using_2007} or phrasal representations~\cite{Hill-etal_2011}.
