
%   i.) Describe all the steps of a Deep Learning FL system
%      
%

Feature location systems retrieve all the relevant program elements
for a developer query, commonly based on a change task. In the {\em
  indexing} step, the systems commonly construct a model of the
software, at the granularity of either classes or methods, based on
the natural language embedded in identifiers and comments. Then in the
{\em retrieval} step, given a user query, feature location systems
retrieve all of the relevant program elements with high similarity to
the query.

A feature location system based on deep learning creates a improved
representation of the system during {\em indexing} stage, while
the {\em retrieval} stage is generally the same.




%  ii.) Describe optimal values for parameters (like window size and feature vector length)
%	*use paper by Poshyvanyk for hints on this
%


%  iii.) What are the expected advantages over its counterparts (n-grams, V-DO, MRF)?
%       *speed and retrieval effectiveness, i suppose
%

%  Show an example of semantic similarity for a project


%  iv.) Other advantages of deep learning?
%       *e.g. query recommendation based on semantic similarity from deep learning

