
%   i.) Describe all the steps of a Deep Learning FL system
%      
%

Feature location systems retrieve a ranked list of program elements
(e.g. methods, classes) for a developer query, based on an index of
terms extracted from the source code. In the {\em indexing} phase,
feature location systems commonly construct a model of the software,
at the granularity of either classes or methods, based on the natural
language embedded in identifiers and comments. In the {\em retrieval}
phase, given a natural language query, feature location systems use
the model to retrieve all of the relevant program elements with high
similarity to the query.

A feature location system based on deep learning creates a contextual
representation of the natural language terms embedded in the source
code, which includes influence from terms preceding and following each
term, relative to their distance from the term. Intuitively, in the
context of source code, these models incoporate mutual influence
between terms in the same method, while terms that occur the same
statement are closer in distance and influence each other more
heavily. Deep learning is based on a multi-stage neural network,
consisting of several hidden layers in addition to an input and output
layers.  The hidden layers serve to capture the context of each
encountered term. Recent advances in this area have stemmed from the
use of novel neural network architectures, includin recurrent neural
networks that connect the hidden layers back to the input layer, among
other strategies. The systems are trained using backpropagation and
gradient descent, techniques common to many neural network based
models. 

Strategies such as word stemming, which reduces terms to their base
form (e.g. X->X) and stop word removal are commonly performed in all
natural language contexts and are commonly leveraged before the neural
networks learning phase. Additional steps are required for source
code, such as identifier splitting that ensures camel case identifiers
are divided into their constituent parts, are also part of most
feature location workflows.




During the retrieval step of feature location, the deep learning model for a
specific code provides a per term representation for the index. Terms in
the user query are located in the index. 

% HOW DOES THIS WORK???

Cosine similarity between the words in a query and each program
element is a common metric that can be used to produce a ranked list
of program elements, which are presented to the developer in
descending order. Figure X shows a diagram of all of the common steps
in a deep learning feature location system.








%  ii.) Describe optimal values for parameters (like window size and feature vector length)
%	*use paper by Poshyvanyk for hints on this
%


%  iii.) What are the expected advantages over its counterparts (n-grams, V-DO, MRF)?
%       *speed and retrieval effectiveness, i suppose
%



%  Show an example of semantic similarity for a project
%  idea: use the 5 poorest performing queries in argouml



%  iv.) Other advantages of deep learning?
%       *e.g. query recommendation based on semantic similarity from deep learning

